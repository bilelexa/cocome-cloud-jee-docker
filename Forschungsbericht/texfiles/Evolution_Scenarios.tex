\chapter{Evolution Scenarios}
We implemented distinct evolution scenarios covering the categories adaptive and perfective
evolution. Corrective evolution is not considered in the scenarios as this merely refers to fixing design or implementation issues.

\section{Evolution Scenarios of the Hybrid Cloud-based Variant}
This section introduces the two evolution scenarios of the hybrid cloud-based variant of
CoCoME.
\subsection{Setting up a Docker environment}
~\\The CoCoME company must reduce IT administration costs but frequent updates to the enterprise and store software are necessary to continuously improve the entire system. As a consequence, IT staff need to update the system components as soon as a new software version is released. An Operations Team member has to get access to the actual server in order to undeploy the old version and replace it with the new one. This is time consuming and expensive as the updates have to be done manually.\\
Therefore, a Docker version is elaborated to simplify the administration process. As soon as a new software version of CoCoME is ready for delivery, the Development Team wrap it into a Docker Image. This Image can be automatically deployed to the destination server according to the principle of Continuous Deployment (CD) \cite{olsson2012climbing}. 




\subsection{Adding a Mobile App}
~\\After successfully adding a Pick-up Shop, the CoCoME company stays competitive with other online shop vendors (such as Amazon). In times of smartphones, customer do not only want to buy exclusively goods from their home computers. Purchasing goods 'on the way' comes more and more into fashion. This raises the idea to create a second sales channel next to the existing Pick-up Shop in the CoCoME system. As a consequence, more customers can be attracted to gain a larger share of the market. 
\\
The customer can order and pay by using the app. The delivery process is similar to the Pick-up Shop: The goods are delivered to a pick-up place (i.e. a store) of her/his choice, for example in the neighbourhood or the way to work.
By introducing the Mobile App as a multi OS application, the CoCoME system has to face various quality issues such as privacy, security and reliability. Also the performance of the whole application can be affected if many customers order via the app.




\newpage

\section{Evolution Scenarios of the Microservice-based Variant}
This section introduces the evolution scenario of the Microservice-based variant of
CoCoME.
\subsection{Defining different Microservices}
After years of growth of the sales figures, the CoCoME company is thinking about steps to keep this trend. During the first meetings, they figured out, there should be a growth in income when they are establishing more branches. Later on, it became clear that the management system used so far would struggle to manage that situation. In consequence he CoCoME management decided to rebuild that system.\\
In their specification they mentioned at first, that the frontend should be similar to the old one and provide the same functionality. They decided to build an decentralized management system, which provides them a rapid registration of sales, and use different services which should be be able to duplicate them self, e.g. one project to manage the branches with an instance for each branch. In that moment a present computer scientist explained the concept of microservices to the managers.\\
Consequently they decided to re-engineer the given system and split it into microservices since it fulfills all their requirements. 

	
	
